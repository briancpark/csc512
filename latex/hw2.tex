\documentclass[11pt]{article}
\usepackage{csc512}

%%% For FAS
\usepackage{tikz}
\usetikzlibrary{automata, positioning, arrows}

%%%%%%%%%%%%%%%%%%%% name/id
\rfoot{\small Brian Park | 200190057}


%%%%%%%%%%%%%%%%%%%% Course/HW info
\newcommand*{\instr}{Xu Liu}
\newcommand*{\term}{Fall 2022}
\newcommand*{\coursenum}{CSC 512}
\newcommand*{\coursename}{Compiler Construction}
\newcommand*{\hwnum}{2}

\rhead{\LARGE   \fontfamily{lmdh}\selectfont	HW \hwnum}

\lfoot{\small \coursenum, \term, HW \hwnum}

%%%%%%%%%%%%%%%%%%%%%%%%%%%%%% Document Start %%%%%%%%%%%%%%%%%
\begin{document}

%%%%%%%%%%%%%%%%%%%%%%%%%%%%%%%%%%%%%%%%%%%%%%%%%%%%%%%%%%%%%%%%%%%%%%%%%%%%%%%%%%%%%%%%
% Question 1
%%%%%%%%%%%%%%%%%%%%%%%%%%%%%%%%%%%%%%%%%%%%%%%%%%%%%%%%%%%%%%%%%%%%%%%%%%%%%%%%%%%%%%%%
\section*{1}
The following grammar is not suitable for a top-down predictive parser. Identify the problem and correct it by rewriting the grammar. Show that your new grammar satisfies the $LL(1)$ condition.

$$L \rightarrow Ra | Qba$$
$$R \rightarrow aba | caba | Rbc$$
$$Q \rightarrow bbc | bc$$

\begin{Answer}
There is left recursion in the nonterminal $R$. We can correct it as follows:

$$L \rightarrow Ra | Qba$$
$$R \rightarrow abaT | cabaT$$
$$T \rightarrow bcT | \epsilon$$
$$Q \rightarrow bZ$$
$$Z \rightarrow bc | c$$

Now the solution has no left recursion. We must also formally prove that the grammar satisfies the $LL(1)$ grammar s.t. $A \rightarrow \alpha$ and $A \rightarrow \beta$ implies $FIRST^+(\alpha) \cap FIRST^+(\beta) = \varnothing$:

For $L, R, T, Q, Z$:
$$FIRST^+(Ra) \cap FIRST^+(Qba) = \varnothing$$
$$FIRST^+(abaT) \cap FIRST^+(cabaT) = \varnothing$$
$$FIRST^+(bcT) \cap FIRST^+(\epsilon) = \varnothing$$
$$FIRST^+(bZ) = \varnothing$$
$$FIRST^+(bc) \cap FIRST^+(c) = \varnothing$$

Simplified:
$$a, c \cap b = \varnothing$$
$$a \cap c = \varnothing$$
$$b \cap \epsilon = \varnothing$$
$$b \cap c = \varnothing$$
\end{Answer}

\newpage

%%%%%%%%%%%%%%%%%%%%%%%%%%%%%%%%%%%%%%%%%%%%%%%%%%%%%%%%%%%%%%%%%%%%%%%%%%%%%%%%%%%%%%%%
% Question 2
%%%%%%%%%%%%%%%%%%%%%%%%%%%%%%%%%%%%%%%%%%%%%%%%%%%%%%%%%%%%%%%%%%%%%%%%%%%%%%%%%%%%%%%%
\section*{2}
Consider the following grammar:
$$A \rightarrow Ba$$
$$B \rightarrow dab | Cb$$
$$C \rightarrow cB$$
$$C \rightarrow Ac$$

\begin{Parts}
	\Part Does this grammar has left recursions? If you believe so, try to rewrite the grammar to remove left recursions; the new grammar should describe the same set of expressions as the original grammar does.
	\begin{Answer}
		Yes, there is left recursion. From $A \rightarrow Ba, B \rightarrow Cb, C \rightarrow Ac$
		
		We can fix this grammar:
		$$A \rightarrow dabaD | cBbaD$$
		$$B \rightarrow dab | Cb$$
		$$C \rightarrow cB | Ac$$
		$$D \rightarrow cbaD | \epsilon$$
	\end{Answer}
	\Part Is the original grammar a $LL(1)$ grammar? Justify your answer.
	\begin{Answer}
		The original grammar has a left recursion, thus it doesn't satisfy $LL(1)$. Even after eliminating left recursion in $a$, the grammar still doesn't satisfy $LL(1)$ due to disjoint sets.
		
		For $A, B, C, D$:
		$$FIRST^+(dabaD) \cap FIRST^+(cBbaD) = \varnothing$$
		$$FIRST^+(dab) \cap FIRST^+(Cb) = \varnothing$$
		$$FIRST^+(cB) \cap FIRST^+(Ac) = \varnothing$$
		$$FIRST^+(cbaD) \cap FIRST^+(\epsilon) = \varnothing$$

		Simplified:
		$$d \cap c = \varnothing$$
		$$d \cap c, d \neq \varnothing$$
		$$c \cap c, d \neq \varnothing$$
		$$c \cap \epsilon = \varnothing$$
	\end{Answer}
\end{Parts}
	
\newpage

%%%%%%%%%%%%%%%%%%%%%%%%%%%%%%%%%%%%%%%%%%%%%%%%%%%%%%%%%%%%%%%%%%%%%%%%%%%%%%%%%%%%%%%%
% Question 3
%%%%%%%%%%%%%%%%%%%%%%%%%%%%%%%%%%%%%%%%%%%%%%%%%%%%%%%%%%%%%%%%%%%%%%%%%%%%%%%%%%%%%%%%
\section*{3}
Write a grammar to describe all binary numbers that are multiples of four. \\

\textit{Note: 0 is a multiple of 4.
The binary number is unsigned (i.e., it contains no sign bit).
It should not contain leading 0s. For instance 00100 is not legal, but 100 is.}

\begin{Answer}
	$$N \rightarrow 0 | 1 M$$
	$$M \rightarrow 1M | 0M | 00$$
	
	Multiples of 4 in binary are represented with trailing two zeros, \verb|00|. The grammar becomes ambiguous, so it will have to look ahead.
\end{Answer}
\newpage


\end{document}